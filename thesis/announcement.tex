\documentclass{article}

\input{project-info}

\usepackage{graphicx}
\usepackage{amsmath}
\usepackage{float}
\usepackage{listings}
\usepackage{xcolor}
\usepackage[margin=0.7in]{geometry}
\usepackage{changepage}
\usepackage[hidelinks]{hyperref}

\colorlet{grey}{gray!120}

\pagenumbering{gobble}

\begin{document}

\begin{center}

\includegraphics[width=0.3\textwidth]{media/ci-logo.png}\\

\hfill\break

\LARGE
\textbf{\color{grey}Computer Science Master Thesis Presentation}\\

\hfill\break

\Large
{\bf \thesistitle}\\{StatGenerative Model}

\vspace{5mm}

\large
{\bf \studentname}\\ {Abdul Quadir Owais, Muhammed}

\vspace{5mm}

\large
\textit{Examination Committee:}\\[1mm]
{\bf Dr. Micheal Soltys} (Advisor), {\bf
Dr. Jason Isaacs}\\

\end{center}

\begin{adjustwidth}{1in}{1in}
\textit{\bf Abstract:}\\

\vspace{2mm}

\normalsize
\noindent
This thesis investigates synthetic data generation for structured tabular
datasets by comparing two contrasting approaches: Conditional Tabular
GAN (CTGAN) and Adaptive Kernel Density Estimation (AKDE). Using the
Dry Beans dataset, we evaluate how well each model reproduces real data
distributions, preserves feature relationships, and supports downstream
analysis.

A unified, reproducible pipeline is developed to generate and evaluate synthetic data using both statistical and visual methods. Quantitative
assessment includes KS-statistics, Wasserstein distance, coverage, variance
ratios, and correlation differences. Qualitative evaluation applies overlays,
correlation heatmaps, pair plots, and PCA projections.
\end{adjustwidth}

\hfill\break

\begin{center}

\LARGE
{\bf 3:00pm, Tuesday, December 2\textsuperscript{nd}, 2025\\
Online via Zoom}\\

\vspace{4mm}

\large
\href{https://csuci.zoom.us/j/87621689334}{\textbf{Join via Zoom: \texttt{https://csuci.zoom.us/j/87621689334}}}\\

\vspace{15mm}

\large
{\bf All students and faculty are invited}\\

\hfill\break

\small
{\it An Academic Affairs Event}

\end{center}

\end{document}
